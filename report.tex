\documentclass[12pt]{article}
    \usepackage[top=1.5cm,bottom=1.5cm,left=2cm,right=2cm, a4paper]{geometry}
    \usepackage{amsmath}
    \usepackage{listings}
    \usepackage{color}
    \usepackage[colorlinks=true,linkcolor=blue]{hyperref}
    \usepackage{framed}
    \usepackage{titling}
\definecolor{codegreen}{rgb}{0,0.6,0}
\definecolor{codegray}{rgb}{0.5,0.5,0.5}
\definecolor{codepurple}{rgb}{0.58,0,0.82}
\definecolor{backcolour}{rgb}{0.95,0.95,0.92}
\lstdefinestyle{mystyle}{
    backgroundcolor=\color{backcolour},
    commentstyle=\color{codegreen},
    keywordstyle=\color{magenta},
    numberstyle=\tiny\color{codegray},
    stringstyle=\color{codepurple},
    basicstyle=\ttfamily\small,
    breakatwhitespace=false,
    breaklines=false,
    captionpos=b,
    keepspaces=true,
    numbers=left,
    numbersep=5pt,
    showspaces=false,
    showstringspaces=false,
    showtabs=false,
    tabsize=2
}
\lstset{style=mystyle}
\renewcommand{\familydefault}{\sfdefault}
\setlength{\droptitle}{-5em}

\begin{document}
    \title{\textbf{Assignment \#1 - System Calls - REPORT}}
    \author{Name: Nguyen Le Huy - ID: 1611290}
    \maketitle
    \section{Adding a new System Call}
        \subsection{Preparation}
            \textbf{Setup Virtual Machine:} You can use Virtual Box or VMware.\\
            You can choose a version of Ubuntu image at \url{https://www.osboxes.org/ubuntu/}\\
            \textbf{Install packages:} You have to install nessesary packages to compile and install kernel.
            \begin{lstlisting}[language=bash]
$ sudo apt-get update
$ sudo apt-get install build-essential kernel-package 
$ sudo apt-get install libncurses5-dev openssl libssl-dev\end{lstlisting}
            \textbf{QUESTIONS:} Why do we need to install kernel-package?\\
            \textbf{ANSWER:} Because kernel-package helps automate the routine steps required 
            to compile and install a custom kernel.\\
            \textbf{Download and unpack kernel source:} Version 4.4.56 is recommended:
            \begin{lstlisting}[language=bash]
$ $ mkdir ~/kernelbuild && cd ~/kernelbuild
$ wget https://cdn.kernel.org/pub/linux/kernel/v4.x/linux-4.4.56.tar.xz
$ tar -xJf linux-4.4.56.tar.xz\end{lstlisting}
            \textbf{QUESTION:} Why do we have to use another kernel source from the 
            server such as\\ \texttt{http://www.kernel.org}, 
            can we compile the local kernel on the running OS directly?\\
            \textbf{ANSWER:} Because local kernel doesn't have source files (.c), so we can't compile it.
        \subsection{Configuration}
        Copy content of configuration file:
        \begin{lstlisting}[language=bash]
$ cp /boot/config-4.x.x-generic ~/kernelbuild/[kernel directory]/.config\end{lstlisting}
        \textit{Note:} run \texttt{uname -r} to see 4.x.x-generic. 
        [kernel directory] will be \texttt{linux-4.4.56}.\\
        \textbf{Important:} You must rename your kernel version, run terminal 
        at [kernel directory] with:
        \begin{lstlisting}[language=bash]
$ make nconfig // or make menuconfig\end{lstlisting}
        Go to \texttt{"General setup"}, access to 
        \texttt{"(-ARCH) Local version - append to kernel release"}, 
        then enter \texttt{".MSSV"} with MSSV is your student ID. Press \texttt{"F6"} to save 
        and \texttt{"F9"} to quit.\\
        If you can't use \texttt{"make nconfig"}, run:
        \begin{lstlisting}[language=bash]
$ nano .config\end{lstlisting}
        then add your MSSV to \texttt{CONFIG\_LOCALVERSION=".MSSV"} and save the file.\\
        Goto arch/x86/entry/syscalls, add these to both file in this directory:\\
        In \texttt{syscall\_32.tbl}:
        \begin{lstlisting}[language=bash]
[number]    i386    procmem         sys_procmem\end{lstlisting}
        \textbf{QUESTION:} What is the meaning of other parts, i.e. i386, procmem, and \texttt{sys\_procmem}?\\
        \textbf{ANSWER:} \texttt{i386} is the application binary interface refers to the compiled binaries in machine code, 
        \texttt{procmem} is the system call's name, \texttt{sys\_procmem} is the entry point where system call is executed.\\
        In \texttt{syscall\_64.tbl}:
        \begin{lstlisting}[language=bash]
[number]    x32     procmem         sys_procmem\end{lstlisting}
        Open the file \texttt{include/linux/syscalls.h} and add the following lines to the end of this file:
        \begin{lstlisting}[language=C++]
struct proc_segs;
asmlinkage long sys_procmem(int pid, struct proc_segs *info);\end{lstlisting}
        Our system call will be implemented in \texttt{arch/x86/kernel} directory in a source file 
        \texttt{sys\_procmem.c}.\\
        \textbf{QUESTION:} What is the meaning of each line above?\\
        \textbf{ANSWER:} Define \texttt{proc\_segs} struct and \texttt{sys\_procmem} system call entry point, 
        \texttt{sys\_procmem} takes process ID and the address of variable type \texttt{struct proc\_segs}, 
        finds memory information of that process and store them to that variable, \texttt{asmlinkage} tells 
        the compiler to look on the CPU stack for the function parameters, instead of registers.
    \section{System Call Implementation}
    \subsection{Test System Call using modules}
    Before write down the system call in \texttt{sys\_procmem.c}, you should test it first using 
    kernel modules of the current kernel in your virtual machine.\\
    Let's create a file called \texttt{"test.c"}:
    \begin{lstlisting}[language=C++]
#include <linux/module.h>
#include <linux/kernel.h>
#include <linux/init.h>
#include <linux/sched.h>
int init_module(void) {
    /*Put your code here to test*/
}
void cleanup_module(void) {
    printk(KERN_INFO "Bye.\n");
}\end{lstlisting}
    Create a Makefile:
    \begin{lstlisting}
obj-m   += test.o
all:
    make -C /lib/modules/$(shell uname -r)/build M=$(PWD) modules
clean:
    make -C /lib/modules/$(shell uname -r)/build M=$(PWD) clean\end{lstlisting}
    Then run these lines in terminal in directory contains those files:
    \begin{lstlisting}[language=bash]
$ make
$ sudo insmod test.ko
$ sudo rmmod test
$ tail /var/log/syslog\end{lstlisting}
    The 4th line will show you the result of function \texttt{init\_module} and \texttt{cleanup\_module}.
    \subsection{Implementation}
        To implement this system call, use 
        \texttt{task\_struct} and \texttt{mm\_struct} defined in \texttt{include/linux/sched.h}:
        \begin{lstlisting}[language=C++]
struct task_struct {
    struct mm_struct *mm, *active_mm;
    pid_t pid;
    ...
};
struct mm_struct {
    unsigned long start_code, end_code, start_data, end_data;
    unsigned long start_brk, brk, start_stack;
    ...
};      \end{lstlisting}
        Add these to file \texttt{arch/x86/kernel/sys\_procmem.c}:
        \begin{lstlisting}[language=C++]
#include <linux/linkage.h>
#include <linux/sched.h>
struct proc_segs {
    unsigned long mssv; /*Your student ID*/
    unsigned long start_code, end_code, start_data, end_data;
    unsigned long start_heap, end_heap, start_stack;
};
asmlinkage long sys_procmem(int pid, struct proc_segs *info) {
    // TODO: Implement the system call
}       \end{lstlisting}
        In the same folder of \texttt{sys\_procmem.c}, add this line to the end of \texttt{Makefile}:
        \begin{lstlisting}
obj-y       += sys_procmem.o\end{lstlisting}
    \section{Compilation and Installation Process}
        \subsection{Compilation}
            In [kernel directory], run:
            \begin{lstlisting}
$ make -j 4
$ make -j 4 modules\end{lstlisting}
            \textbf{QUESTION:} What is the meaning of these two stages, namely \texttt{"make"} 
            and \texttt{"make modules"}?\\
            \textbf{ANSWER:} \texttt{"make modules"} just compiles the modules. \texttt{"make"} 
            compiles the kernel and create \texttt{vmlinuz} (kernel image which will be uncompressed and 
            loaded into memory by GRUB or any boot loader you use).
        \subsection{Installation}
            In [kernel directory], run:
            \begin{lstlisting}
$ sudo make -j 4 modules_install
$ sudo make -j 4 install
$ sudo reboot\end{lstlisting}
        Hold \texttt{"shift"} key while booting, choose \texttt{"Advanced..."} then choose kernel.\\
        Check kernel version using \texttt{uname -r}.
        \subsection{Testing}
            Use these program to test:
            \begin{lstlisting}[language=C]
#include <sys/syscall.h>
#include <stdio.h>
#include <unistd.h>
int main() {
    long sysvalue;
    unsigned long info[100];
    sysvalue = syscall (377, 1, info);
    printf("My MSSV: %lu\n", info[0]);
    return 0;
}           \end{lstlisting}
            \textbf{QUESTION:} Why this program could indicate whether our system call works or not?\\
            \textbf{ANSWER:} Because if it works, it will print your student ID.
    \section{Making API for System Call}
        Create \texttt{wrapper} directory:
        \begin{lstlisting}
$ mkdir ~/wrapper\end{lstlisting}
        Put this code in \texttt{procmem.h} in \texttt{wrapper}:
        \begin{lstlisting}[language=C++]
#ifndef _PROC_MEM_H_
#define _PROC_MEM_H_
#include <unistd.h>
#include <sys/types.h>
struct proc_segs {
    unsigned long mssv; /*Your student ID*/
    unsigned long start_code, end_code, start_data, end_data;
    unsigned long start_heap, end_heap, start_stack;
};
long procmem(pid_t pid, struct proc_segs *info);
#endif // _PROC_MEM_H_\end{lstlisting}
        \textbf{QUESTION:} Why we have to re-define proc\_segs struct while we 
        have already defined it inside the kernel?\\
        \textbf{ANSWER:} Because we can't get that struct in kernel directly, the compiler won't know 
        where to find that struct if we don't re-define it.\\
        Put this code in \texttt{procmem.c} in \texttt{wrapper}:
        \begin{lstlisting}[language=C++]
#include "procmem.h"
#include <linux/kernel.h>
#include <sys/syscall.h>
long procmem(pid_t pid, struct proc_segs *info) {
    // TODO: implement the wrapper here.
}\end{lstlisting}
        Copy our header file to header directory of our system and compile as a shared object:
        \begin{lstlisting}
$ cd ~/wrapper
$ sudo cp procmem.h /usr/include
$ gcc -shared -fpic procmem.c -o libprocmem.so\end{lstlisting}
        If the compilation ends successfully:
        \begin{lstlisting}
$ sudo cp libprocmem.so /usr/lib\end{lstlisting}
        \textbf{QUESTION:} Why root privilege (e.g. adding sudo before the cp command) is 
        required to copy the header file to \texttt{/usr/include}?\\
        \textbf{ANSWER:} Because only root user can write files and folders in \texttt{/usr}\\
        \textbf{QUESTION:} Why we must put -share and -fpic option into gcc command?\\
        \textbf{ANSWER:} Because -shared creates a shared library which is loaded by programs when 
        they start. Using -fpic option usually generates smaller and faster code, but will have 
        platform-dependent limitations, such as the number of globally visible symbols or the 
        size of the code.\\
\end{document}
    
    